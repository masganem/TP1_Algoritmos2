\documentclass[10pt,a4paper]{article}

% Modern font setup
\usepackage{fontspec}
\usepackage{unicode-math}

% Font selection
\setmainfont{Fira Sans}
\setsansfont{Fira Sans}
\setmonofont[
  UprightFont   = *,
  BoldFont      = *-Bold,
  ItalicFont    = *,    % <-- force regular to be used as “italic”
  BoldItalicFont= *-Bold        % <-- or omit if you don’t need Bold Italic
]{Fira Code}
% \setmathfont{Fira Math} % trouble with some symbols...

% Language and layout
\usepackage[utf8]{inputenc} % optional with XeLaTeX, but safe
\usepackage[brazil]{babel}
\usepackage{geometry}
\geometry{margin=2.5cm}

% Good to keep
\usepackage{amsmath}
\usepackage{graphicx}
\usepackage{url}
\usepackage{xcolor}
\usepackage{outlines}

% Bibliography
\usepackage[backend=biber,style=authoryear]{biblatex}
\renewcommand*{\nameyeardelim}{\addcomma\space}

\title{Relatório -- Trabalho Prático 1\vspace{0.25cm}\\ \Large Algoritmos 2}
\author{{\bfseries Marcelo Ganem, Rafael Paniago, Pedro Loures }\vspace{0.075cm} \\ \texttt{\{marceloganem,rafaelpaniago,pedroloures\}@dcc.ufmg.br} \vspace{0.15cm}\\ Universidade Federal de Minas Gerais}
\date{\today}

\newcommand{\note}[1]{
    \vspace{0.3cm}
    \colorbox{blue!30}{
            \begin{minipage}{0.4\textwidth}
		    \ttfamily \footnotesize
               #1
            \end{minipage}
        }
    \vspace{0.3cm}
}

\setlength{\columnsep}{1cm} % or whatever you prefer
\begin{document}

\makeatletter
% Title page, optional
% \begin{titlepage}
  % \centering
  % \vspace*{5.5cm}
    % {\Large \bfseries Above title \par}
    % \vspace{1cm}
    % {\Huge \bfseries \@title \par}
  % \vspace{1cm}
  % \vspace{0.5cm}
  % {\Large \@author \par}
  % \vspace{0.5cm}
  % \vfill {\large \today\par}
% \end{titlepage}

\twocolumn
\maketitle
% \begin{abstract}
    % This is the abstract.
% \end{abstract}


\section{Especificação do Problema}
\label{specs}

O código que acompanha este relatório implementa o \textbf{armazenamento e consulta de pontos geográficos}, bem como suas informações correspondentes -- pertinentes aos bares cadastrados na Prefeitura de Belo Horizonte e ao festival \textit{Comida di Buteco} -- em uma \textbf{árvore} \textbf{K-dimensional}.

\subsection{Dados de entrada}
\label{raw-data}

A base de dados primária para a implementação é o relatório em formato \texttt{.csv} de localização das atividades econômicas cadastradas no município de Belo Horizonte disponibilizada mensalmente pela Secretaria Municipal da Fazenda da PBH\footnote{\url{https://dados.pbh.gov.br/dataset/atividades-economicas1}}.

Adicionalmente, dados \textbf{não-estruturados} do festival \textit{Comida di Buteco}, disponíveis na lista de participantes do evento\footnote{\url{https://comidadibuteco.com.br/butecos/belo-horizonte}}, são necessários para a execução do exercício extra.

Por fim, definimos a entrada do usuário como um \textbf{retângulo} sob o espaço bidimensional representado pelo mapa de Belo Horizonte.

\subsection{Árvores K-Dimensionais}
\label{kd-trees}
O enunciado propõe a organização dos dados obtidos conforme latitude e longitude em uma árvore K-dimensional. Essa estrutura permite realizar buscas em intervalos ortogonais\footnote{Do inglês \textit{orthogonal range search.}} em tempo $O(\sqrt{n} + k)$ no caso bidimensional (para uma árvore com $n$ nós e retornando $k$ resultados) dado um processo de construção de custo $O(n \log n).  $\footnote{Na implementação ótima. A implementação utilizada aqui constrói a árvore em $O(n \log^2 n)$.}

\subsection{Ferramentas}
\label{tools}
A implementação do trabalho na linguagem Python requer a OpenStreetMaps API\footnote{\url{https://www.openstreetmap.org/}} para obtenção de coordenadas geográficas correspondentes aos endereços dos dados primários e a biblioteca \textit{dash-leaflet}\footnote{\url{https://www.dash-leaflet.com/}} para a construção da interface gráfica. Todas as ferramentas adicionais utilizadas são explicitamente justificadas na Seção \ref{implementation}.


\subsection{Requisitos}
\label{requirements}
A tarefa consiste em implementar uma interface web que permita ao usuário selecionar uma área retangular no mapa de Belo Horizonte e, por meio da busca intervalar na árvore K-dimensional, filtrar entre os restaurantes em uma determinada tabela.

\subsubsection{Requisitos adicionais (\textit{Comida di Buteco})}
Como uma tarefa opcional, dados do festival Comida di Buteco podem ser cruzados com os dados da tarefa principal, aumentando a interface web com um \textit{pop-up} dinâmico que exibe informações\footnote{Vale observar que as informações se limitam às imagens disponíveis na galeria de restaurantes -- isso porque as páginas dos estabelecimentos específicos não estão disponíveis desde (pelo menos) 03/06/2025.} sobre o prato concorrente de restaurantes participantes.

\section{Implementação}
\label{implementation}

\subsection{Processamento de dados}
\label{data}

\subsubsection{Filtragem}
\label{filtering}
Um primeiro filtro seleciona somente as entradas de CNAE 5611201.0, 5611204.0 ou 5611205.0, correspondentes a bares e restaurantes. Então, descartamos as colunas excedentes, mantendo: nome, endereço -- formatado como uma única \textit{string}, data de início das atividades e um campo \textit{booleano} condicional à existência de um alvará emitido pela prefeitura.


\subsubsection{Geolocalização}
\label{geolocalization}
Aumentamos os dados filtrados com coordenadas geográficas correspondentes ao endereço de cada bar utilizando a biblioteca \texttt{geopy}\footnote{\url{https://geopy.readthedocs.io/en/stable/}}. Assim, definimos os valores a serem utilizados para ordenação na árvore K-dimensional. 

\subsubsection{Deduplicação}
\label{deduplication}
Deduplicamos os dados, tendo em vista a duplicação extensiva presente no conjunto original, em função do campo \texttt{NOME} da entrada (não processado).

\subsubsection{Coleta e casamento (\textit{Comida di Buteco})}
\label{cdb-scraping-and-matching}
Os dados foram extraídos por meio de \textit{web scraping} utilizando a biblioteca \texttt{bs4}\footnote{\url{https://pypi.org/project/beautifulsoup4/}}. O casamento entre os dados do \textit{Comida di Buteco} e da PBH é feito com base em uma representação normalizada da forma \texttt{rua\#número}, e então desambiguado com base na contagem de palavras em comum no campo \texttt{NOME} -- também normalizado. Por \textit{normalização}, nos referimos à remoção de acentos, caracteres não alfanúmericos e palavras e sequências irrelevantes.


\subsection{Árvores K-Dimensionais}
\label{kd-trees-impl}
A implementação da árvore para busca intervalar é feita em C++ e Python. Utilizamos a biblioteca \texttt{pybind11}\footnote{\url{https://github.com/pybind/pybind11}} para compilar o código da classe que implementa a árvore K-dimensional em C++, incluindo um método construtor $O(n \log^2$ $ n)$ e um método de consulta $O(\sqrt{n} + k)$. A implementação em Pyhton apresenta métodos equivalentes que invocam o método na classe original, e é incluída por conveniência.

\subsection{Interface Web}
\label{web}
O aplicativo \textit{dash-leaflet} implementado apresenta um mapa e uma tabela que filtra constantemente os restaurantes sob a seleção retangular atual -- que pode ser iniciada pelo usuário clicando no botão correspondente no canto superior esquerdo. Adicionalmente, restaurantes participantes do \textit{Comida di Buteco} aparecem em destaque nas primeiras posições da tabela.

Selecionar um restaurante na tabela ativa um marcador no mapa e centraliza a visualização atual no restaurante selecionado. Passar o mouse (\textit{hover}) no marcador de um restaurante participante do \textit{Comida di Buteco} ativa um \textit{pop-up} mostrando informações do restaurante e uma imagem do prato concorrente.

\section{Considerações}
\label{considerations}
\begin{outline}[enumerate]
        \1 A geolocalização de dados e a posterior filtragem por dados geolocalizados para \textit{display} no mapa causa uma perda de aproximadamente 3.000 entradas.
        \1 Os critérios de casamento descritos na Subseção \ref{cdb-scraping-and-matching} casam 78 dos 124 restaurantes participantes com os dados deduplicados não geolocalizados, e 51 com os dados filtrados por geolocalização.
        \1 A árvore é construída sempre que o servidor é incializado. Para otimização, uma ideia é armazenar a estrutura ordenada em disco -- essa solução é mais escalável para um número de entradas fora dos limites específicos deste trabalho.
\end{outline}

\end{document}
